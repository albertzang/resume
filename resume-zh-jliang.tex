% !TEX program = xelatex
\documentclass[letterpaper]{article}
\usepackage{calc}
\usepackage{xeCJK}
\usepackage{xcolor}
\usepackage{xparse}
\usepackage{textpos}
\usepackage{setspace}
\usepackage{geometry}
\usepackage{fontawesome5}
\usepackage[T1]{fontenc}
\usepackage[nomessages]{fp}
\usepackage[utopia]{mathdesign}
\usepackage[pdfborder={0 0 0}]{hyperref}
\def\fontxs{\fontsize{8pt}{10pt}\selectfont}
\def\fontsm{\fontsize{10pt}{12pt}\selectfont}
\def\fontmd{\fontsize{11pt}{13pt}\selectfont}
\def\fontlg{\fontsize{12pt}{14pt}\selectfont}
\def\fontxl{\fontsize{30pt}{32pt}\selectfont}
\def\indentsize{5pt}
\definecolor{light-gray}{gray}{0.9}

% flushpage %
\NewDocumentCommand{\flushpage}{}{
  \newpage\FPeval{\vcursor}{0}
}

% gridspace %
% #1 = vertical space in unit of grid
\NewDocumentCommand{\gridspace}{m}{
  \FPeval{\vcursor}{\vcursor+#1}
}

% grid %
% #1 = keep vcursor
% #2 = width
% #3 = height
% #4 = hpos
% #5 = content
% #6 = color
\NewDocumentCommand{\grid}{smmmmo}{
  \def\gridbox{\parbox[c][#3\TPVertModule][c]{#2\TPHorizModule}{#5}}
  \begin{textblock}{#2} (#4,\vcursor)
    \IfValueTF{#6}{\colorbox{#6}{\gridbox}}{\gridbox}
  \end{textblock}
  \IfBooleanF{#1}{\gridspace{#3}}
}

% letterhead %
% #1 = company name
% #2 = street address
% #3 = city address
% #4 = country address
% #5 = letter subject
\NewDocumentCommand{\letterhead}{mmmmm}{
  \gridspace{1}
  \grid{6}{2}{0}{\fontsm\textit{\today}}
  \gridspace{1}
  \grid{6}{2}{0}{\fontsm\textsc{#1}}
  \gridspace{0.5}
  \grid{6}{2}{0}{\fontsm#2}
  \grid{6}{2}{0}{\fontsm#3}
  \grid{6}{2}{0}{\fontsm#4}
  \gridspace{1}
  \grid{6}{2}{0}{\fontsm\textbf{\underline{#5}}}
  \gridspace{1}
}

% headerfull %
% #1 = name
% #2 = city
% #3 = country
% #4 = phone
% #5 = email
\NewDocumentCommand{\headerfull}{mmmmm}{
  \grid*{6}{4}{0}{\fontxl\textbf{\textsc{#1}}}
  \grid{6}{4}{6}{\fontxs\hspace{\fill}\LaTeX}
  \grid*{6}{2}{0}{\fontsm\texttt{#2}}
  \grid{6}{2}{6}{\fontsm\hspace{\fill}\texttt{#4}}
  \grid*{6}{2}{0}{\fontsm\texttt{#3}}
  \grid{6}{2}{6}{\fontsm\hspace{\fill}\texttt{#5}}
  \grid{12}{1}{0}{\rule{12\TPHorizModule}{0.5pt}}
}

% headershort %
% #1 = name
% #2 = phone
% #3 = email
\NewDocumentCommand{\headershort}{mmm}{
  \grid*{6}{4}{0}{\fontxl\textbf{\textsc{#1}}}
  \grid{6}{2}{6}{\fontsm\hspace{\fill}\texttt{#2}}
  \grid{6}{2}{6}{\fontsm\hspace{\fill}\texttt{#3}}
  \grid{12}{1}{0}{\rule{12\TPHorizModule}{0.5pt}}
}

% sectiontitle %
% #1 = icon
% #2 = title
\NewDocumentCommand{\sectiontitle}{om}{
  \gridspace{1}
  \grid{12}{2}{0}{\fontlg\hspace{\indentsize}\IfValueT{#1}{\faicon{#1}\hspace{\indentsize}}\textsc{#2}}[light-gray]
  \gridspace{1}
}

% subsectiontitle %
% #1 = subtitle
% #2 = sidenote
\NewDocumentCommand{\subsectiontitle}{mO{}}{
  \grid*{6}{2}{0}{\fontmd\hspace{\indentsize}\textbf{#1}}
  \grid{6}{2}{6}{\fontsm\hspace{\fill}\textit{#2}\hspace*{\indentsize}}
}

% griditem %
% #1 = item
\NewDocumentCommand{\griditem}{m}{
  \grid{12}{2}{0}{\fontsm\hspace{\indentsize}#1}
}

% doublegriditem %
% #1 = bold top-left
% #2 = italic bottom-left
% #3 = normal top-right
% #4 = italic bottom-right
\NewDocumentCommand{\doublegriditem}{mmO{}O{}}{
  \grid*{6}{1.5}{0}{\fontsm\hspace{\indentsize}\textbf{#1}}
  \grid{6}{1.5}{6}{\fontsm\hspace{\fill}#3\hspace*{\indentsize}}
  \grid*{6}{1.5}{0}{\fontsm\hspace{\indentsize}\textit{#2}}
  \grid{6}{1.5}{6}{\fontsm\hspace{\fill}\textit{#4}\hspace*{\indentsize}}
}

\setCJKmainfont[Path=./fonts/, Extension=.ttf, BoldFont=MicrosoftYaHei-Bold, AutoFakeSlant=true]{MicrosoftYaHei}
\setCJKsansfont[Path=./fonts/, Extension=.ttf, BoldFont=MicrosoftYaHei-Bold, AutoFakeSlant=true]{MicrosoftYaHei}
\setCJKmonofont[Path=./fonts/, Extension=.ttf, BoldFont=MicrosoftYaHei-Bold, AutoFakeSlant=true]{MicrosoftYaHei}
\def\name{梁家恩}
\def\phone{\href{tel:+17789298864}{(+1)778.929.8864}}
\def\email{\href{mailto:liangjiaen2017@gmail.com}{liangjiaen2017@gmail.com}}
\def\home{\href{https://maps.app.goo.gl/5JedGP1KRUeaAWsi9}{Vancouver BC, Canada}}
\begin{document}
\grid*{60}{9}{60}{\hfill\fontxl\textbf{\name}}
\grid{60}{3}{0}{\fontsm\faIcon{phone} \phone}
\grid{60}{3}{0}{\fontsm\faIcon{at} \email}
\grid{60}{3}{0}{\fontsm\faIcon{home} \home}
\grid{120}{2}{0}{\rule{120\TPHorizModule}{0.5pt}}[][0pt]
\spacer{1}
\grid{120}{4}{0}{\fontmd\faIcon{graduation-cap} 教育背景}[lightgray]
\spacer{1}
\grid{120}{3}{0}{\fontsm{人类学\ 硕士}}
\grid{120}{3}{0}{\fontsm\textbf{中山大学}\hfill\fontxs{2012年9月 --- 2014年6月 | 中国,广州}}
\grid{120}{3}{0}{\hspace{3\TPHorizModule}\fontsm{主修课程:文化人类学理论、中国人类学史、人类学研究法、海外华人研究}}
\grid{120}{3}{0}{\fontsm{新闻学\ 学士}}
\grid{120}{3}{0}{\fontsm\textbf{广东财经大学}\hfill\fontxs{2008年9月 --- 2012年6月 | 中国,广州}}
\grid{120}{3}{0}{\hspace{3\TPHorizModule}\fontsm{主修课程:古代汉语、基础写作、新闻学概论、编辑学概论、图书学、中国历代文学作品选读}}
\spacer{1}
\grid{120}{4}{0}{\fontmd\faIcon{certificate} 培训与证书}[lightgray]
\spacer{1}
\grid{120}{3}{0}{\fontsm{幼教助教证书ECEA Certificate}
\hfill\fontsm{Early Childhood Educator Registry,British Columbia}}
\grid{120}{3}{0}{\fontsm{蒙特梭利助教证书AMI Assistant Certificate}
\hfill\fontsm{Montessori Training Centre of British Columbia}}
\grid{120}{3}{0}{\fontsm{儿童急救证书First Aid \& CPR/AED Level B}
\hfill\fontsm{Canadian Red Cross}}
\grid{120}{3}{0}{\fontsm{高等学校教师岗前培训证书}
\hfill\fontsm{广东省高等学校师资培训中心}}
\grid{120}{3}{0}{\hspace{3\TPHorizModule}\fontsm{培训课程:高等教育学、高等教育心理学、高等学校教师职业道德修养、高等教育法规概论}}
\spacer{1}
\grid{120}{4}{0}{\fontmd\faIcon{cog} 专业技能}[lightgray]
\spacer{1}
\grid{120}{9}{0}{\fontsm
    \textbullet\ 具有丰富的校园工作经验,善于与学生打交道\\
    \textbullet\ 具备期刊出版领域的工作经验,培养专业的编辑技能\\
    \textbullet\ 熟练掌握国语、粤语、英语
}
\spacer{1}
\grid{120}{4}{0}{\fontmd\faIcon{suitcase} 工作经历}[lightgray]
\spacer{1}
\grid{120}{3}{0}{\fontsm{市场专员 | ETIG易格留学移民公司}
\hfill\fontsm{\faIcon{map-marker-alt}加拿大温哥华\ \ \ \ \faIcon[regular]{calendar-alt}2020 --- 2022}}
\grid{120}{9}{0}{\fontsm
    \textbullet\ 负责营销计划和方案的起草工作\\
    \textbullet\ 负责公司业务的推广和咨询工作\\
    \textbullet\ 协助持牌移民顾问办理留学和移民的申请
}
\grid{120}{3}{0}{\fontsm{编辑助理 | 上海交通大学中国公益发展研究院}
\hfill\fontsm{\faIcon{map-marker-alt}中国上海\ \ \ \ \faIcon[regular]{calendar-alt}2015 --- 2017}}
\grid{120}{9}{0}{\fontsm
    \textbullet\ 支持总编完成学术期刊《中国第三部门研究》的采编任务\\
    \textbullet\ 管理学术资源使用权限,协助开展学术会议,保持与学者的紧密联系\\
    \textbullet\ 参与非政府组织相关的项目和会议,促进慈善事业和志愿服务
}
\grid{120}{3}{0}{\fontsm{招聘专员 | 中山大学南方学院}
\hfill\fontsm{\faIcon{map-marker-alt}中国广州\ \ \ \ \faIcon[regular]{calendar-alt}2014 --- 2015}}
\grid{120}{9}{0}{\fontsm
    \textbullet\ 开展全国性以及港台教师招聘和引进计划\\
    \textbullet\ 协助员工各类人事手续办理\\
    \textbullet\ 维护和招聘渠道的关系
}
\spacer{1}
\grid{120}{4}{0}{\fontmd\faIcon{hand-holding-heart} 志愿者经历}[lightgray]
\spacer{1}
\grid{120}{3}{0}{\fontsm{有间童书馆\ 馆主 (自创自营)}
\hfill\fontsm{\faIcon{map-marker-alt}加拿大温哥华\ \ \ \ \faIcon[regular]{calendar-alt}2020 --- 至今}}
\grid{120}{9}{0}{\fontsm
    \textbullet\ 为大温哥华地区的新妈妈和新移民提供信息、资源和政府资助项目\\
    \textbullet\ 共享儿童中英文绘本、教育视频、学习资料等,组织线上读书会\\
    \textbullet\ 定期举办讲座,主题涵盖加拿大儿童福利、双语情境下华裔儿童的中文学习等
}
\grid{120}{3}{0}{\fontsm{温哥华图书馆\ 图书馆冠军项目 Library Champion Project}
\hfill\fontsm{\faIcon{map-marker-alt}加拿大温哥华\ \ \ \ \faIcon[regular]{calendar-alt}2024 --- 至今}}
\grid{120}{6}{0}{\fontsm
    \textbullet\ 向新移民推广图书馆资源\\
    \textbullet\ 协助图书馆开展宣传活动
}
\grid{120}{3}{0}{\fontsm{S.U.C.C.E.S.S\ 接待和咨询服务}
\hfill\fontsm{\faIcon{map-marker-alt}加拿大温哥华\ \ \ \ \faIcon[regular]{calendar-alt}2018 --- 2019}}
\grid{120}{6}{0}{\fontsm
    \textbullet\ 接待新移民,提供咨询服务\\
    \textbullet\ 协助各类新移民活动的开展
}
\end{document}

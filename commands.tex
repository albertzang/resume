\pagestyle{empty}
\setstretch{0}
\setlength{\parindent}{0pt}
\setlength{\fboxrule}{0pt}
\setlength{\fboxsep}{0pt}
\FPeval{\vcursor}{0}
% letterpaper is of 8.5 inch width * 11 inch height
% 0.5 inch horizontal margins and 0.5 inch vertical margins
% 2 * 160 grids, each grid is 3.75 inch in width and 1/16 inch in height
\geometry{letterpaper, total={7.5in, 10in}, hcentering, noheadfoot}
\TPGrid[0.5in, 0.5in]{2}{160}

\def\fontsm{\fontsize{2\TPVertModule}{3\TPVertModule}\selectfont}
\def\fontmd{\fontsize{2.5\TPVertModule}{4\TPVertModule}\selectfont}
\def\fontlg{\fontsize{6\TPVertModule}{8\TPVertModule}\selectfont}
\def\indentsize{0.02\TPHorizModule}
\definecolor{light-gray}{gray}{0.85}

% flushpage %
\NewDocumentCommand{\flushpage}{}{\newpage\FPeval{\vcursor}{0}}

% gridspace %
% #1 = vertical space in unit of grid
\NewDocumentCommand{\gridspace}{m}{\FPeval{\vcursor}{\vcursor+#1}}

% grid %
% #1 = keep vcursor
% #2 = width
% #3 = height
% #4 = hpos
% #5 = content
% #6 = color (optional)
\NewDocumentCommand{\grid}{smmmmo}{
  \def\gridbox{\parbox[c][#3\TPVertModule]{#2\TPHorizModule}{#5}}
  \begin{textblock}{#2} (#4,\vcursor)
    \IfValueTF{#6}{\colorbox{#6}{\gridbox}}{\gridbox}
  \end{textblock}
  \IfBooleanF{#1}{\gridspace{#3}}
}

% headerfull %
% #1 = name
% #2 = motto
% #3 = phone
% #4 = email
% #5 = linkedin
% #6 = home
\NewDocumentCommand{\headerfull}{mmmmmm}{
  \grid*{1}{8}{1}{\hspace{\fill}\fontlg\textbf{\textsc{#1}}}
  \grid{1}{4}{0}{\hspace{\indentsize}\faIcon{phone}\hspace{\indentsize}\fontmd\texttt{#3}}
  \grid{1}{4}{0}{\hspace{\indentsize}\faIcon{envelope}\hspace{\indentsize}\fontmd\texttt{#4}}
  \grid*{1}{8}{1}{\hspace{\fill}\fontsm\textit{#2}}
  \grid{1}{4}{0}{\hspace{\indentsize}\faIcon{linkedin-in}\hspace{\indentsize}\fontmd\texttt{#5}}
  \grid{1}{4}{0}{\hspace{\indentsize}\faIcon{home}\hspace{\indentsize}\fontmd\texttt{#6}}
  \grid{2}{1}{0}{\rule{2\TPHorizModule}{0.5pt}}
}

% headershort %
% #1 = name
% #2 = phone
% #3 = email
\NewDocumentCommand{\headershort}{mmm}{
  \grid*{1}{8}{0}{\fontlg\textbf{\textsc{#1}}}
  \grid{1}{4}{1}{\fontmd\hspace{\fill}\texttt{#2}}
  \grid{1}{4}{1}{\fontmd\hspace{\fill}\texttt{#3}}
  \grid{2}{1}{0}{\rule{2\TPHorizModule}{0.5pt}}
}

% sectiontitle %
% #1 = fontawesome icon
% #2 = title
\NewDocumentCommand{\sectiontitle}{mm}{
  \gridspace{1}
  \grid{2}{4}{0}{\fontmd\hspace{\indentsize}\faIcon{#1}\hspace{\indentsize}\textsc{#2}}[light-gray]
  \gridspace{1}
}

% subsectiontitle %
% #1 = subtitle
% #2 = sidenote (optional, default empty)
\NewDocumentCommand{\subsectiontitle}{mO{}}{
  \grid{2}{4}{0}{\fontmd\hspace{\indentsize}#1\fontmd\hspace{\fill}\textit{#2}\hspace*{\indentsize}}
}

% degree %
% #1 = bold top-left
% #2 = normal top-right
% #3 = italic bottom-left
% #4 = italic bottom-right
\NewDocumentCommand{\degree}{mmmm}{
  \grid{2}{3}{0}{\fontsm\hspace{\indentsize}\textbf{#1}\hspace{\fill}#2\hspace*{\indentsize}}
  \grid{2}{3}{0}{\fontsm\hspace{\indentsize}\textit{#3}\hspace{\fill}\textit{#4}\hspace*{\indentsize}}
}

% gridpar %
% #1 = height
% #2 = paragraph
\NewDocumentCommand{\gridpar}{mm}{
  \begin{textblock}{2} (0,\vcursor)
    \setstretch{1}
    \centering
    \parbox[c][#1\TPVertModule]{2\TPHorizModule-10pt}{\fontsm#2}
  \end{textblock}
  \gridspace{#1}
}
